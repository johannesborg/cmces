\documentclass{article}
\usepackage{graphicx} % Required for inserting images

\title{mcspaper}
\author{johpe18 }
\date{February 2025}

\begin{document}

\maketitle

\section{Introduction}


In this paper we will describe an algorithm which can find all connected maximum common subgraphs of a set of an arbitrary number of simple, labeled graphs. The algorithm we describe here will employ a sub-algorithm for finding maximal common subgraphs between pairs of graphs. In theory any algorithm which can exactly find all connected maximal common subgraphs between two graphs could be used. In this paper we consider the case where the sub-algorithm used is the well known modular graph product approach coupled with a modified version of the Bron-Kerbosch algorithm for finding maximal cliques. We will also describe a heuristic for pruning the modular graph product, while preserving the exactness of the algorithm. It turns out that the ordering of the input graphs has a big impact on the runtime of our algorithm, so we will describe a heuristic for finding a good ordering based on any given similarity measure, where such a similarity measure is a function that maps two graphs to a real number.  
\\
\\
First we will provide some necessary definitions. Then we will describe how to find maximal common subgraphs between a pair of graphs by finding cliques in the modular product of the graphs. We will describe a modified version of the Bron-Kerbosch algorithm, which in its most basic form finds all maximal cliques in a graph. We will describe a heuristic for removing edges from modular product graphs, which in practice speeds up the algorithm. We will describe the algorithm itself, and finally we will describe our heuristic for ordering the input graphs. 
\\
\\

In the results section we will present experimental results from running our algorithm on different benchmarks and molecular datasets.


\section{Definitions}


\section{Maximal common subgraphs of two graphs}

\subsection{Maximal cliques in the modular product}

\subsection{Modified Bron-Kerbosch}

\subsection{Pruning modular product graphs}


\section{Maximum common subgraphs of more than two graphs}

\subsection{Algorithm}

\subsection{Ordering input graphs}

\subsubsection{Similarity measures}



\section{Results}



\section{Discussion}




\end{document}
